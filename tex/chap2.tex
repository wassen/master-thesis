\chapter{対象と手法}\chaplab{Materials and methods}
この章では、本研究における手法を述べる。まず、1節では、本研究に用いた運転行動データについての詳細を、2節では、運転行動データから抽出した特徴を、3説では、特徴点の推定方法について述べる。
\section{運転行動データ}
本研究では、名古屋大学が収集した運転行動データを用いた。このデータでは、10名の被験者が名古屋高速道路を走行した際の様々な運転行動が記録されている。この内、本研究に用いた運転行動は、以下のようになっている。
\begin{itemize}
 \item 車線変更ラベル
 \item アクセルの踏み込み圧
 \item ブレーキの踏み込み圧
 \item ハンドルの操舵角
 \item 自車両中心からの周辺車両との相対位置
 \item 周辺車両との相対距離
\end{itemize}
車線変更ラベルは、直進、右車線変更、左車線変更の3値で与えられる。また、周辺車両との相対位置、相対速度は車載のレーザーレンジファインダーを用いて計測されており、全後方に対して最大100mの車両を検知している。また、相対位置、相対速度は、道路平面に対して車両進行方向と、その垂直方向の2次元となっている。(y軸の定義とかしちゃう?)
また、被験者たちは高速道路を走行する際に、車線変更をして前方車両を抜かし、車両を抜かし終わったら走行車線へ戻るように指示されている。
また、車線変更開始は、「車載カメラによる動画内で白線が左右に動き始めたとき」で、車線変更終了は、「白線の動きが止まったとき」と定義されている。これらのデータのサンプリングレートは、10Hzである。車線変更は、各被験者平均して、右車線変更左車線変更ともに、合計30回程度行われており、合計341回の右車線変更、335回の左車線変更が行われていた。
\section{特徴抽出}
右車線変更をするタイミングを推定したい。
この運転行動データを用いて、車線変更のタイミングの推定に有用だと思われる特徴の抽出を行った。
先行車両との相対距離と相対速度、右後方の車両との相対距離と相対速度、また、加速度を考慮したTTCにの逆数(iTTC\_2nd)を用いた。
これは、被験者たちは、以下のように行動していたからである。はじめは前方の車両との距離が縮まったとき、右車線変更しても追い越し車線で衝突しなさそうなら右車線変更を開始する。?
この時の
車線変更開始時点での特徴量の動きを見るために、車線変更開始時点から10秒前の時点から車線変更開始までの値を取り出した。

\subsection{可視化}
ひとつの車線変更トライアルに対して、プロットを行ったもの
車線変更開始時点でのプロット
車線変更開始10秒前でのプロット
回転している感じを表したかった?
PCAに持って行くまでの流れ?
変化量が重要であると考えられる。そのため、n-1の時刻での各特徴との4次元特徴を作成した。
PCAをかけて、
寄与率
ガウス分布で近似したもの
0.5secにした理由?実用上?
