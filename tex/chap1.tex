\chapter{はじめに}
\section{背景}
近年,自動車の普及台数は世界的に急速に増加\cite{who}しており,先進国,発展途上国問わず,自動車は人々の生活に欠かせないものとなっていくものと考えられる.一方で,自動車は未だ人々にとって十分に安全な乗り物とは言えない.世界では,交通事故により年間あたり120万人もの人が命を落としているという現状があり\cite{who},交通事故発生の対応策が必要であると考えられる.
\par
交通事故の原因として,道路や交通の状況が考えられるが,人的要因もその一つである.速度違反,飲酒運転等の危険運転が交通事故に繋がるのは言うまでもないことであるが,脇見等の些細な不注意でも,危険運転同様に大きな事故に繋がる可能性がある.このような人的錯誤は,事前の注意喚起,教育による習慣形成,懲罰の設定等の制限による予防を危険運転ほどには望めないと考えられる.
\par
そのため,不注意により発生する事故を事前に察知し,予防するような運転者を支援するシステムを導入するという方法が効果的であると考えられる.このようなシステムは,Advanced Driver Assistance System(ADAS)とよばれ,未来の乗り物として期待されている.前述したような運転者の支援には,運転者の行動予測が不可欠であり,このような予測システムを実現するには運転行動のモデル化が必要である.
\par
運転行動のモデリングに関する研究は,大きく分けて巨視的な交通の流れを見るものと,微視的な個々の車両の行動に着目するものの2つに分かれている
\cite{flow}
が,本論文においては,「運転行動のモデル」は,後者の微視的なものを指すものとする.
% 運転行動について,事故率,運転行動についての解析がなされた,運転支援には予測が不可欠
運転行動のモデリングを行った研究は数多く存在しているが,本論文では高速道路における車線変更をモデル化したものを扱うものとするが,本論文にて示す手法は他の運転行動に対しても応用可能である.高速道路において車線変更を予測するタスクを扱った論文として,Kumar\cite{kumar}らの論文がある.この論文では,車線からの横方向の距離と,ハンドル角とその時間変化を特徴として,Suppot Vector Machineで分類した結果をBayesian Filterで平滑化することで,白線を越える1秒前までに高い精度,再現率で車線変更開始を予測できた,としている.また,Morris\cite{morris}らは,多数のセンサを活用しRVMで車線変更タイミングを予測したところ,運転者の顔の角度から車線変更開始を予測できた,としている.
\par
これらの既存の研究では,「車線変更を開始した時刻」の定義が,「車両が白線をまたいだときとなっており,運転者が車線変更開始に向けてハンドルを切った後にその検出に成功している.また,明確にいつ車線変更を開始するかの時刻を得られるわけではなく,その時点において車線変更を開始しそうかどうかを知ることしかできない.その為,実際に何秒後に車線変更するのか,ということや,運転者がハンドルを切ることになった根本的な原因は何にあるのかという,より進んだ車線変更開始モデルの構築が課題である.
\par
運転行動,特に車線変更というタスクにおいては,自車両の動きだけでなく,周辺車両との関わりあいが大きな影響を及ぼすと考えられ,周辺車両情報を有効に活用することでより良い運転行動モデルを構築することが期待できるが,周辺車両とのかかわり合いを考慮した運転行動モデルについての研究は非常に少ない\cite{survey}.周辺車両との関わりあいの指標の一つとして,車両との衝突までの時間,Time to Collision(TTC)が挙げられる.森田\cite{Morita}らは,ブレーキを開始するまでのタイミングのの判別を行い,先行車両の大きさの変化率がブレーキのタイミングに寄与しており,TTCでは判別困難であることを突き止めた,としている.
\\
\section{研究目的}
本研究では,車線変更開始までの時刻を推定するための手法を開発することを第一の目標とした.また,有用であると考えられるにもかかわらず周辺車両との関係から特徴を抽出することはできていないため,周辺車両との自車両との関係性と車線変更開始までの時刻に如何な関係性があるのかを探ることも目標とした.
\\
\section{本論文の構成}
本論文は以下のような構成を取る.まず,\chapref{Materials_and_Visualize}では,本研究の対象とした運転行動データの詳細と,車線変更開始前の運転行動データを可視化することにより,車線変更開始時刻の推定に有用な特徴や,本研究の手法を提案するに至るまでの考察を述べる.次に,\chapref{methods}では,車線変更開始タイミングの推定方法についての詳細や,実際の計算上で行った諸手続きについて述べる.\chapref{result},\chapref{discussion}では,推定結果をもとに,得られた知見と,改善点を議論し,\chapref{conclusion}につなげる.


% より頑健?たかだか車線変更を開始してから白線をまたぐまでの時間しか最大で予測できないが,それ以上前から推定できる
% 車線変更開始がどのような原因で引き起こされているのか,
% 車線変更が開始されてから,車線を超える前に警告を出す,というめんでは不十分であると考えられる.
% 自車両じゃなくてもいけんじゃねこれ
