\chapter{はじめに}
 \section{背景}
% 自動車の歴史
運転支援システムや,自動運転技術などが近年盛んに研究されており,未来の乗り物として期待されている.運転支援システムでは,人間の運転行動を深く知る必要があり,自動運転技術では,運転行動を機械的に記述する必要がある.このため,運転行動のモデル化が必要である.引用なし...
\par
運転行動のモデリングに関する研究は,大きく分けて巨視的な交通の流れを見るものと,微視的な個々の車両の行動に着目するものの2つに分かれている
\cite{flow}
が,本論文においては,「運転行動のモデル」は,後者の微視的なものを対象とする.ADASにつかうなら,まあ,行動の予測っしょ.てきな
運転行動について,事故率,運転行動についての解析がなされた,運転支援には予測が不可欠
さて,予測する自動車の運転行動の対象としては,
大別して,交差点における右左折等の行動を予測するもの,高速道路において車線変更をするか否かをモデル化したものの2種類に分けることができる.
高い精度で
また,後出しジャンケンみたいなことをしたいわけじゃなくて,時刻を推定したい.
車線変更が開始されてから,車線を超える前に警告を出す,というめんでは不十分であると考えられる?
車線変更の相互関係を用いて車線変更行動予測をした研究は未だ少なく[あれ],どのような特徴が有用であるのかはまだわかっていない.

\\
\section{研究目的}
そのため本研究では,車線変更開始までのタイミングを推定するための手法を開発することを目標とした.また,運転者の行動と周辺車両との関係性を考慮する?
\\
\section{本論文の構成}
本論文は
% TTCの説明どこでしようか
