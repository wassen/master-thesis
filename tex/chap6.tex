\chapter{結論}\chaplab{conclusion}
本研究では,運転行動のモデリングの課題の一つとして,運転者による車線変更の時刻推定を目標とした.まず,車線変更開始時刻を決定づける要因として,どのような特徴量が関わっているのかを調査した.その結果,自車両の前方車両との相対距離と相対速度が特徴量として有用そうであるということを突き止めた.次に,相対距離,相対速度が車線変更開始前にどのように分布しているのかを調べたところ,データ点は,二次元の正規分布で近似できそうであること,データ点が回転しながら収束しているように見えることがわかった.ここから,回転の情報を保持するために,変位の情報を加えた4次元のベクトルに主成分分析をかけることによって2次元に落とし込んだものを特徴量とした.
\par
この特徴量から車線変更の開始時刻を推定するために,車線変更開始時間の持つ分布を逐次的に更新するベイズ的な手法をもちいた.この手法により得られた分布から,MAP,重み付き平均の両方で時刻を推定したところ,データ点の取得が進むほどに推定時刻が改善されていくことが確認できた.
% 分布として得られるからいいんだっていうやつ入れたい.
