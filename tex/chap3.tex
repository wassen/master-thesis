
\chapter{車線変更時刻推定法}\chaplab{methods}
前章では,車線変更前の前方車両との距離,速度が二次元の正規分布に従っていると考えられることを示した.本章においては,この仮定を用いて,逐次的に車線変更を行う時刻を推定する手法について解説する.まず,では
\section{逐次ベイズ推定}
車線変更を開始するまでの時間を$n$($N$以下の任意の自然数)と置く.自車両と周辺車両の関係についての観測値が与えられていない時,$n$は一様分布であると仮定する.この時事前分布は
\begin{align}
p(n)=1/N
\end{align}
と表される.次に観測値が得られた時,変数$n$に対する確率分布$p(n)$を更新し,
更新する.
ところで?
ここで,二次元の正規分布は,
\begin{align}
\mathcal{N}(\xvec|\muvec,\Sigmavec)=\frac{1}{2\pi\detSigmavec^{1/2}}\exp\left\{-\frac{1}{2}(\xvec-\muvec)^\mathrm{T}\Sigmavec^{-1}(\xvec-\muvec)\right\}
\end{align}
と書ける.ただし,$\muvec$は2次元の平均ベクトル,$\Sigmavec$は$2\times2$の共分散行列,$\detSigmavec$は$\Sigmavec$の行列式である.

$x$の観測値t個の観測値のデータ集合$\mathcal{D}_t=\{x_1,\ldots,x_t\}$

観測値$x$

これらから,
サンプルが得られた時,どのガウス分布に属するのかについて事前分布を仮定し,各時刻の持つガウス分布の尤度を
用いて分布を更新することで車線変更からの時刻推定を試みた.
各時刻におけるnの式は,以下のようにして表される.
10秒前から0.5秒間隔でデータを与え,分布を更新した.
事後分布の重み付き平均をとった時,MAP推定を行った.
今回,$N=10$
