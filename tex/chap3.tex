\chapter{手法}\chaplab{methods}
前章では,車線変更前の前方車両との距離,速度は,車線変更が始まるまでの時間ごとに,二次元の正規分布に従っていると考えることができることを示した.
\section{推定}
車線変更を開始するまでの時間を$t$と置く,新しくデータ点が得られた時,変数$t$に対する確率分布を更新することで車線変更開始までの時間を推定する.まず何もデータが与えられていない時,車線変更開始までの時間を等確率と仮定する.事前分布$p(t)$は
\begin{align}
p(t)=1/20
\end{align}
と表される.
また,ガウス分布は,以下のように表されるため.
\begin{align}
temp
\end{align}
これらから,
サンプルが得られた時,どのガウス分布に属するのかについて事前分布を仮定し,各時刻の持つガウス分布の尤度を
用いて分布を更新することで車線変更からの時刻推定を試みた.
各時刻におけるnの式は,以下のようにして表される.
10秒前から0.5秒間隔でデータを与え,分布を更新した.
事後分布の重み付き平均をとった時,MAP推定を行った.
