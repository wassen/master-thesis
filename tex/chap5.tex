\chapter{考察}\chaplab{discussion}
\section{MAP,重み付き平均について}
本研究の提案手法の評価として,MAP,重み付き平均の両方における推定結果を比較した.重み付き平均では中心部に引っ張られてしまい,大きな誤差は出ないが,完全に推定結果が一致することは少なかった.MAPでは,完全に推定結果が一致することもあれば,全く結果が一致しないこともあった.\chapref{result}でも述べたように,これらの結果は一長一短である.単純にMAP,重み付き平均ではなく,これらの両方を考慮した評価基準を考案することにより,推定結果の改善が図れると考えられる.
\section{推定の失敗について}
提案手法により得られた分布をMAPを用いて推定を行うと,常に「車線変更開始直前である」という推定結果が一定数存在していることがわかった.つまり,運転者の行動によっては,車線変更のタイミング推定が失敗することが考えられる.原因としては,\figref{pca_gauss}に見る通り,各時刻に属する特徴量の正規分布が十分な分離ができていないということがあげられる.また,全時刻で直前と推定されてしまうことから,車線変更開始10秒前から直前まで一貫して同じ相対距離,同じ相対速度をドライバーが維持し続けている,ということも原因として考えられる.これらを根本的に解決する方法としては,より車線変更開始時刻を決定づける特徴量を発見することが必要である.
\par
% まずひとつは,前方車両との関係性についてのより良い特徴量を探ることである.距離と速度から計算できるTTCが車線変更開始時間の推定に有用でないことは,\chapref{Materials_and_Visualize}で述べたとおりだが,
実際の運転行動から,前方車両の距離と速度には法則性がみられた.一つは,等速を維持したまま距離を詰め,そのまま車線変更を開始する,という場合である.これは単純に,前方車両が障害となるため回避行動を行ったと考えられる.もう一つは,30m程度の距離を維持して,減速の後加速しながら車線変更を開始する場合である.ドライバーは一旦車線変更をやめて前方車両の後ろに付き,その後車線変更を開始していた.状況として,追い越し車線に車両がいたため,車線変更を一旦やめている事が考えられる.このような状況では,前方車両だけでなく,追い越し車線の車両についても推定に有用であると考えられる.本研究における将来的な課題は,この追い越し車線の車両について有用な特徴量の表現を探ることである.


% クラスタ
% 少なく見積もって2つのクラスタがあることがわかった.
% これらの動きの違いは運転行動からくるもの
% 分離し,それぞれに有用な特徴

\par
% もうひとつは,前方車両以外に車線変更の時刻決定に及ぼす要因を探ることである.特定の
% 今回,追い越し車線を走行する車両を特徴量としては加えなかったが,実際の運転行動をみると,追い越し車線の車両が通過するのを待ってから右車線変更を開始する,という状況は確かに存在しており,追い越し車線との関係性を特徴量として盛り込むことが有効であると考えられる.
