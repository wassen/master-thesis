\chapter{考察}\chaplab{discussion}
\section{MAP,重み付き平均について}
本研究の提案手法の評価として,MAP,重み付き平均の両方を結果として
重み付き平均では,
中心部に引っ張られてしまう.
\chapref{result}でも述べたように,推定に失敗して,常に車線変更開始直前であると一長一短である.両方を考慮した評価基準
\section{推定の失敗について}
提案手法により得られた分布をMAPを用いて推定を行うと,常に「車線変更開始直前である」という推定結果が一定数存在していることがわかった.ここから,運転者の行動によっては,車線変更のタイミング推定が失敗することが考えられる.原因としては,各時刻に属する特徴量の正規分布が未だ十分な分離ができていないということがあげられる.また,全時刻で直前と推定されてしまうことから,車線変更開始10秒前から直前まで一貫して同じ相対距離,同じ相対速度を維持し続けている,ということも原因として考えられる.これらを根本的に解決する方法としては,より車線変更開始時刻を決定づけるより良い特徴量を発見することが必要である.考えられる新しい特徴量へのアプローチは2つある.
\par
まずひとつは,前方車両との関係性についてのより良い特徴量を探ることである.

TTCが車線変更開始時間の推定に有用でないことは,\chapref{Materials_and_Visualize}で述べたとおりだが,



クラスタ
少なく見積もって2つのクラスタがあることがわかった.
これらの動きの違いは運転行動からくるもの
分離し,それぞれに有用な特徴

\par
もうひとつは,前方車両以外に車線変更の時刻決定に及ぼす要因を探ることである.特定の
今回,追い越し車線を走行する車両を特徴量としては加えなかったが,実際の運転行動をみると,追い越し車線の車両が通過するのを待ってから右車線変更を開始する,という状況は確かに存在しており,追い越し車線との関係性を特徴量として盛り込むことが有効であると考えられる.
