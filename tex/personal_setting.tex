\newcommand{\jdoctitle}{修士論文}
\newcommand{\edoctitle}{Master's Thesis}
\newcommand{\studentnumber}{1551003}  % 学籍番号
\newcommand{\jtitle}{逐次学習による車線変更タイミングの推定}  % 修論の題名
\newcommand{\etitle}{Estimation of Timing of Lane Changing by Sequential Learning}     % 英語の題名
\newcommand{\jauthor}{浅山 和宣}      % 著者名
\newcommand{\eauthor}{Kazuki Asayama} % 英語の著者名
\newcommand{\jdate}{\today}
\newcommand{\edate}{\ifcase\month\or
    January\or February\or March\or April\or May\or June\or
    July\or August\or September\or October\or November\or December\fi
    \space\number\day,\space \number\year}
\newcommand{\keywords}{修士論文,奈良先端科学技術大学院大学}
\newcommand{\ekeywords}{Master's Thesis, NAIST}

\newcommand{\jabstract}{ % 日本語概要
 そのころわたくしは、モリーオ市の博物局に勤めて居りました。
 十八等官でしたから役所のなかでも、ずうっと下の方でしたし俸給(ほうきゅう)もほんのわずかでしたが、
 受持ちが標本の採集や整理で生れ付き好きなことでしたから、わたくしは毎日ずいぶん愉快にはたらきました。
 殊にそのころ、モリーオ市では競馬場を植物園に拵(こしら)え直すというので、
 その景色のいいまわりにアカシヤを植え込んだ広い地面が、切符売場や信号所の建物のついたまま、
 わたくしどもの役所の方へまわって来たものですから、わたくしはすぐ宿直という名前で月賦で買った小さな蓄音器と
 二十枚ばかりのレコードをもって、その番小屋にひとり住むことになりました。
 わたくしはそこの馬を置く場所に板で小さなしきいをつけて一疋の山羊を飼いました。
 毎朝その乳をしぼってつめたいパンをひたしてたべ、それから黒い革のかばんへすこしの書類や雑誌を入れ、
 靴もきれいにみがき、並木のポプラの影法師を大股にわたって市の役所へ出て行くのでした。
}

\newcommand{\eabstract}{ % 英語概要
 Lorem ipsum dolor sit amet, consectetur adipisicing elit,
 sed do eiusmod tempor incididunt ut labore et dolore magna aliqua.
 Ut enim ad minim veniam, quis nostrud exercitation ullamco laboris nisi ut aliquip ex ea commodo consequat.
 Duis aute irure dolor in reprehenderit in voluptate velit esse cillum dolore eu fugiat nulla pariatur.
 Excepteur sint occaecat cupidatat non proident, sunt in culpa qui officia deserunt mollit anim id est laborum.
}
