\newcommand{\jdoctitle}{修士論文}
\newcommand{\edoctitle}{Master's Thesis}
\newcommand{\studentnumber}{1551003}  % 学籍番号
\newcommand{\jtitle}{逐次ベイズ推定による車線変更時刻の推定}  % 修論の題名
\newcommand{\etitle}{Estimation of Timing of Lane Changing by Sequential Bayes Inference}     % 英語の題名
\newcommand{\jauthor}{浅山 和宣}      % 著者名
\newcommand{\eauthor}{Kazuki Asayama} % 英語の著者名
\newcommand{\jdate}{\today}
\newcommand{\edate}{\ifcase\month\or
    January\or February\or March\or April\or May\or June\or
    July\or August\or September\or October\or November\or December\fi
    \space\number\day,\space \number\year}
\newcommand{\keywords}{運転行動,モデリング,車線変更,逐次ベイズ推定,時系列予測}
\newcommand{\ekeywords}{Driving behavior,Modeling,Lane changing,Sequential Bayes Inference,Time series prediction}

\newcommand{\jabstract}{ % 日本語概要
  自動車は世界各国の人々にとって,生活に欠かせない道具となりつつあるが,交通事故で命を落とす人は年間120万人に上る.事故を抑制する手段の一つとして,Advanced Driver Assistance System(ADAS)のような運転を支援するデバイスの研究が近年進められてきてきた.このようなシステムにおいては,運転行動のモデリングによってより進んだ機能が実現できると考えられる.運転行動のモデリングについての研究は数多くなされているが,周辺車両と運転行動の関連性についての研究は未だ少ない.そして,車線変更は周辺車両に大きく影響すると考えられる.そのため,本研究ではモデルの対象を車線変更に絞った.車線変更を予測し高い成果を上げた研究はいくつか存在するものの,何れも,周辺車両との関わりを明らかにしていないこと,ハンドル操作を始めてから推定を行うこと,推定時刻を得られないこと等の課題が残っている.そこで,本研究では,周辺車両との関わりから,車線変更開始までの時刻に分布をおき,逐次ベイズ推定により分布を更新することで車線変更推定を行った.この結果,分布を更新していくことでより誤差の少ない推定結果を得ることができ,周辺車両との関連性を用いて逐次的なアプローチを用いることが車線変更開始の推定に有用であることがわかった.
}

\newcommand{\eabstract}{ % 英語概要
  Motor-vehicles are indispensable for worldly people to support their lives. On the other hand, there is the fact that 1.2 million people are killed by road accidents. Therefore the studies have been conducted on driving support devices like Advanced Driver Assistance System (ADAS) which prevent traffic accidents. The driving model can make such system more intelligent. The driving model is well studied, however, there are few interaction models between surrounding vehicle. Moreover, a lane changing is strongly influenced by surrounding vehicle. Hence we studied about the lane changing model. There are several successful studies on lane changing prediction, however, all have the following tasks: not reveal relations about surrounding vehicles, start prediction after turning the steering, not predict concrete estimation time. In this study, we estimated lane changing using surrounging vehicles by assuming a distribution about starting time from lane changing, then update the distribution by sequential Bayes inference. As a result, updating distribution can acquire result with less estimation error, therefore, this approach is useful to estimate time of lane changing.
}
